\documentclass{beamer}
\usepackage[utf8]{inputenc}

\usetheme{Madrid}
\usecolortheme{default}

%------------------------------------------------------------
%This block of code defines the information to appear in the
%Title page
\title[Cloud Atlas] %optional
{Cloud Atlas}

\subtitle{An LstmEncoder for UHECR AirShowers}

\author[Gianluca Becuzzi, Lucia Papalini] % (optional)
{G. Becuzzi \and L. Papalini}

\date[July 2022] % (optional)
{July 2022}

%End of title page configuration block
%------------------------------------------------------------



%------------------------------------------------------------
%The next block of commands puts the table of contents at the 
%beginning of each section and highlights the current section:

\AtBeginSection[]
{
  \begin{frame}
    \frametitle{Table of Contents}
    \tableofcontents[currentsection]
  \end{frame}
}
%------------------------------------------------------------


\begin{document}

%The next statement creates the title page.
\frame{\titlepage}


%---------------------------------------------------------
%This block of code is for the table of contents after
%the title page
\begin{frame}
\frametitle{Table of Contents}
\tableofcontents
\end{frame}
%---------------------------------------------------------


\section{Introduction}

%---------------------------------------------------------
\begin{frame}{UHECR Airshowers}


\end{frame}

%---------------------------------------------------------


%---------------------------------------------------------
\begin{frame}{Dataset, first glance}
    Description of dataset

\end{frame}

%---------------------------------------------------------

\section{Preprocessing}

%---------------------------------------------------------
\begin{frame}{Split the dataset}
    funky\_dtype e compagnia
\end{frame}

%---------------------------------------------------------
\begin{frame}{Split the dataset}
    non so ma forse seriviranno 2 slide
\end{frame}

%---------------------------------------------------------
\begin{frame}{Data Augmentation}
    Augmentation class e amici
    
\end{frame}



\section{Neural Network building}

%---------------------------------------------------------
\begin{frame}{DataFeeder class}

    
\end{frame}

%---------------------------------------------------------
\begin{frame}{DataFeeder class}

    
\end{frame}

%---------------------------------------------------------
\begin{frame}{Overview on the network}

    
\end{frame}

%---------------------------------------------------------
\begin{frame}{Encoder for time of arrivals}

    
\end{frame}
%---------------------------------------------------------
\begin{frame}{Encoder performance}
i graficini loss accuracy ecc ecc
    
\end{frame}

%---------------------------------------------------------
\begin{frame}{LSTM}
si spiega che cos'è
    
\end{frame}

%---------------------------------------------------------
\begin{frame}{LSTM for the time series}
si fa vedere come abbiamo fatto noi
    
\end{frame}

%---------------------------------------------------------
\begin{frame}{LSTM performance}

    same
\end{frame}

%---------------------------------------------------------
\begin{frame}{Concatente + dense layers}

    
\end{frame}

%---------------------------------------------------------
\begin{frame}{Network's output}

    
\end{frame}

%---------------------------------------------------------
\begin{frame}{Hyperparameters tuning}

    
\end{frame}


%---------------------------------------------------------
\begin{frame}{Whole Network performance}

    
\end{frame}

%---------------------------------------------------------
\begin{frame}{Danke e bibliography}
\centering
Danke Schon

    
\end{frame}




\end{document}
